% BEGIN
\documentclass{article}

% UNICODE
\usepackage[utf8]{inputenc}
\usepackage[T1]{fontenc}
\usepackage{wasysym}
\usepackage{oplotsymbl}

% LAYOUT
\usepackage{calc}
\newlength{\unit}\setlength{\unit}{9.5625pt} % ONE- SIXTY-FOURTH PAGE WIDTH
\newlength{\marginheaderl}\setlength{\marginheaderl}{0\unit}
\newlength{\widthheader}\setlength{\widthheader}{24\unit}
\newlength{\marginl}\setlength{\marginl}{4\unit}
\newlength{\marginr}\setlength{\marginr}{4\unit}
\usepackage[
    paperwidth=612pt,  
    paperheight=792pt, 
    margin=\marginr, 
    top=8\unit, 
    left=\marginheaderl + \widthheader + \marginl, 
    bottom=8\unit, 
    marginparsep=\marginl, 
    marginparwidth=\widthheader
]{geometry}

% PACKAGES
\usepackage[overlay]{textpos}
\usepackage{hyperref}
\usepackage{graphicx}
\usepackage{xcolor}
\usepackage{microtype}
\usepackage{enumitem}
\usepackage{fp}

% SUPPRESS UNDERFULL BOX WARNINGS
% \hbadness=999999

% HYPERLINKS HAVE ODD FORMATTING
\hypersetup{%
  colorlinks=false,% HYPERLINKS WERE RED
  linkbordercolor=red,% THEY HAD A RED BORDER
  pdfborderstyle={/S/U/W 0}% THEY HAD A BORDER AND 1PT UNDERLINE
}

% FONTS
\usepackage{TheanoDidot} % ROMAN
\usepackage{Chivo} % SANS_SERIF
\usepackage{inconsolata} % MONOSPACE
\renewcommand{\familydefault}{\sfdefault}
\newcommand{\defaultfontsize}{\fontsize{\unit}{1\unit}\selectfont}

% FORMATTING
\linespread{1}
\setlength\parindent{0pt}
\reversemarginpar
\pagenumbering{gobble}

% SECTIONS
\newcommand{\sectionskip}{\vspace{4\unit}}
\newcommand{\makeHeader}[1]{
    \leavevmode\marginpar{%
        \ttfamily
        \defaultfontsize
        \textblockcolor{black}%
        \begin{flushright}\leavevmode\leaders\hbox{\space»}\hfill\kern0pt\uppercase{\space#1}\end{flushright}
    }%
}

% DOCUMENT
\begin{document}
    \defaultfontsize
    \raggedright

    \makeHeader{%
        Programming 2\\ 
        \vspace{\baselineskip}
        CPTR.212\\
        Fall 2020 | Southwestern College\\
        Professor Evan Daniel\\
        Evan.Daniel@SCKans.edu\\
        \vspace{\baselineskip}
        Christy 12\\ 
        MWF 10:00AM-10:50AM\\
        Lab F 11:00AM-11:50AM\\
        Office Hours M-F 12:00PM- 1:00PM\\
    }%
    Programming 2 is a course in which we will find that deeper un\-d\-e\-r\-s\-t\-anding of computer science principles allows us to do more progra\-m\-m\-atically.
    
    \vspace{\baselineskip}We will build on the introduction to programming provided in Programming 1.  Now that we've seen how effective code can be, we will develop a rigorous understanding of the mechanisms behind it.  By the course's conclusion, students will be able to develop both command-line and GUI (graphical) software that is compiled (i.e., it will not rely on other programs to do the work).
    
    \vspace{\baselineskip}Using C, we will address topics including memory and how processes use it, object-oriented programming, data structures, and algorithms.  We will also build a toolbox of industry-standard technologies to address issues like portability, versioning, and debugging.

    \sectionskip\makeHeader{Course Catalog Listing}%
    CPTR212/Lecture/A - Programming 2 | Credits 4.00\\
    Object-oriented programming (OOP), data structures (list, stack, queue, tree, graph), and fundamental algorithms will be covered in this class. The primary language used for instruction is C++. Lecture and laboratory Prerequisites: CPTR 110 and 112. Credit 4 hours.

    \sectionskip\makeHeader{Course Deliverables}%
    \vspace{-2\unit}
    \begin{itemize}[leftmargin=*]
        \item Proficiency in C's implementation of programmatic principles, including syntax, memory, objects, data structures, and algorithms.
            \begin{itemize}
                \item\textit{50 minutes in-class per week}
                \item\textit{3 hours 30 minutes out-of-class per week}
            \end{itemize}
        \item Working knowledge of the shell layer, including the terminal application or shell, common C APIs and libraries, and the Unix/Linux environment.
            \begin{itemize}
                \item\textit{50 minutes in-class per week}
                \item\textit{2 hours and 20 minutes out-of-class per week}
            \end{itemize}
        \item Familiarity with setting up a development environment for original production, including tools such as GCC (GNU Compiler Collection), git, github.com, Windows Subsystem for Linux (WSL), and the VSCode integrated development environment (IDE).  
            \begin{itemize}
                \item\textit{50 minutes in-class per week}
                \item\textit{1 hour and 10 minutes out-of-class per week}
            \end{itemize}
        \item Experience verbalizing programmatic issues, including verbal analysis of code and researching solutions through other resources (e.g. man pages, web resources, our textbook).
            \begin{itemize}
                \item\textit{50 minutes in-class per week}
                \item\textit{2 hours and 20 minutes out-of-class per week}
            \end{itemize}
    \end{itemize}
    \vspace{-1\unit}

    \sectionskip\makeHeader{Attendance}%
    Attendance can be in-person or through Zoom.  Attendance will be recorded, but there is no penalty for absences (including "total absence"; neither in-person nor on Zoom).  
    
    \vspace{\baselineskip}If a student is absent on a day an assignment is due, they are required to set up a meeting with the instructor to be held within one academic week of returning to class.  It is their responsibility to set up this meeting, to be prepared to present their work, and to allot ten minutes to discuss each assignment.  If they do not do so, they will receive a grade of 0 for the assignment.  
    
    \sectionskip\makeHeader{Assessment}%
    Assignments will be assessed through dialogue between student and instructor.  Assessment criteria will be not the outcome of the assignment (whether the assignment is impressive, or works well), but on the student's demonstrated understanding of the assignment in discussion.

    \vspace{\baselineskip}Each assignment will have a grade recorded between 0 to 1 (e.g. "0.5"; "0.875").  At the end of the semester, these grades will be averaged with a perfect grade of 1 and multiplied by 100 (e.g., if the student's average grade is 0.75, their final grade will be (0.75 + 1.00) / 2 * 100, or 87.5\%).  

    \sectionskip\makeHeader{Ethics}%
    This course is a space where we acknowledge and value the agency of each of our peers and the diversity of our community.  To be consistent with those values, all communication within the course --- whether in the form of spoken word, submitted assignments, online communication, or any other form --- must allow all other individuals in the course to freely participate.
    
    \vspace{\baselineskip}Students whose verbal communication prevents or precludes others from being part of our community or discourse will be asked to leave the course meeting, and can be made subject to further academic discipline. Submitted work that prevents or precludes others from being part of our community or discourse will not be assessed, with no points awarded.  Note that plagiarism is detrimental to our discourse and therefore falls under this category.
    
    \sectionskip\makeHeader{Southwestern College\\ Builder Community Health Pledge}%
    Our pledge to shared responsibility and community health: 
    \begin{itemize}
        \item I will know and check for COVID-19 symptoms daily
        \item I will stay in my residence when I have a temperature about 100 degrees Fahrenheit
        \item I will practice frequent hand-washing
        \item I will maintain 6-feet of social distance wherever possible
        \item I will wear a mask in buildings when outside of my residence room or individual office
        \item I will avoid large social gatherings
        \item I will limit my personal travel as possible and necessary
        \item I will follow and abide by directions and guidelines of college and Public Health officials related to the need to identify and contact trace any possible COVID-19 cases or exposures
    \end{itemize}
    I understand that the health of everyone in our Builder community is dependent upon shared responsibility, and I will do my part to help protect my community.  I will demonstrate care and respect for others.  This is the Builder Way.  
    
    \sectionskip\makeHeader{Southwestern College\\ Disability Services Statement}%
    Students in this course who have a disability that might prevent them from fully demonstrating their academic abilities should contact Steve Kramer, Disability Services Coordinator as soon as possible to initiate disability verification and discuss accommodations.  Steve Kramer’s office is located in the basement of the Christy Administration building, through the double glass doors.  He can be reached at (620) 229-6307 or at disability.services@sckans.edu. In the event of his absence, Arthur Smith, Disability Services Coordinator for Professional Studies will respond. Please also consult the \href{https://www.sckans.edu/student-services/student-success-and-retention/disability-services/}{\color{blue}\bfseries Disability Services Webpage}.
    
    \vspace{\baselineskip}Southwestern College has an office specifically designed to help you, the student, with any issues you may have.  We will guide you to the help you need and have the expertise to make difficult problems more manageable.  You can reach us at Student.Success@sckans.edu, you will see us around campus or you can come by the office in the basement of Christy through the double glass doors.  We are here for your success.

    \sectionskip\makeHeader{Required Resources}%
    \href{https://www.manning.com/books/modern-c}{\color{blue}\bfseries Modern C} (2019), Jens Gustedt\\
    This will be our main text for the course.

    \vspace{\baselineskip}\href{https://ubuntu.com/wsl}{\color{blue}\bfseries Ubuntu on Windows Subsystem for Linux (WSL)}\\
    WSL is in extremely active development, and is quickly becoming a standard for developers working on Windows.  It will be our chosen tool to interact with the command line.

    \vspace{\baselineskip}\href{https://git-scm.com/}{\color{blue}\bfseries Git} and \href{https://github.com}{\color{blue}\bfseries github.com}\\
    We will use git in conjunction with github.com for submitting assignments.  Both are ubiquitous in professional programming environments.  

    \vspace{\baselineskip}\href{https://code.visualstudio.com}{\color{blue}\bfseries VSCode (Visual Studio Code)}\\
    Analytics have shown that VSCode is currently the most-used IDE for programmers in the world.  VSCode offers

    \vspace{\baselineskip}\textbf{Your college-issued laptop}\\
    Development environments vary considerably across different platforms; in order to address operating system and architecture issues collectively, it is required that all students use their school-issued laptops.  

    \sectionskip\makeHeader{Recommended Resources}%
    \href{https://en.wikipedia.org/wiki/The_C_Programming_Language}{\color{blue}\textbf{The C Programming Language, Second Edition}} (1988), Brian Kernighan and Dennis Ritchie\\
    This is not only co-written by one of the designers of C (Ritchie), it is one of the most famous programming books of all time.  It is short (only about 150 pages) and focuses on the most important parts of the language.

    \vspace{\baselineskip}\href{https://stackexchange.com}{\color{blue}\textbf{StackExchange.com}}\\
    This is \textit{the} site for searching for answers and explanations to random bugs you're coming across.
\end{document}